\documentclass{article}
\usepackage{graphicx} 
\usepackage{hyperref}

\title{Design patterns for mobile apps}
\author{Ernesto Garcia}
\date{February 4, 2024}

\begin{document}

\maketitle

\section{Design patterns}

\begin{enumerate}
    \item Navigation Drawer: Allows users to access different sections of the app by sliding out a side panel from the edge of the screen. Ideal for apps with multiple sections or functionalities.
    \item Tabs: Organizes app content into multiple horizontal views, typically located at the top of the screen. Useful for switching between different categories of content or functionalities.
    \item Bottom Navigation Bar: Places the app's main sections or features into icons at the bottom of the screen. Facilitates one-handed navigation on large mobile devices.
    \item Floating Action Button: A prominent circular button that performs the primary action on the current screen. Used for high-priority actions like creating a new item or starting a new task.
    \item Cards: Presents information in rectangular cards with related content, such as an image, a title, and a description. Helps organize and display content clearly and concisely.
    \item Master-Detail: Displays a list of items in a main view (master) and the details of the selected item in a secondary view (detail). Suitable for showing detailed information about individual items in a collection.
    \item Search Bar: Allows users to search for specific content within the app. Enhances usability by making search and quick access to relevant information easier.
    \item Onboarding: Guides users through key features of the app at the beginning of their experience. Provides instructions and tips to help users become familiar with the app.
    \item Pull-to-Refresh: Enables users to refresh the content of a list or view by swiping down. Provides an intuitive way to update content without the need for additional buttons.
    \item Modal: Displays a modal popup window that requires user attention before continuing. Useful for confirmations, alerts, and critical actions that require an immediate response.
\end{enumerate}

\section{References}
\begin{itemize}
    \item \href{https://www.designrush.com/best-designs/apps/trends/mobile-design-patterns}{DesignRush: Mobile Design Patterns}
    \item \href{https://keepcoding.io/blog/patrones-de-diseno-en-interfaces-moviles/}{KeepCoding Blog: Patrones de Diseño en Interfaces Móviles}
\end{itemize}

\end{document}
